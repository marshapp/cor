\documentclass[11pt]{article}
\usepackage{graphicx} % Required for inserting images
\usepackage[top=2.5cm, bottom=2.5cm, left=2.5cm, right=2.5cm]{geometry}
\usepackage[T1]{fontenc}
\usepackage{hyperref}
\usepackage[utf8]{inputenc}
\usepackage{multirow}
\usepackage{booktabs}
\usepackage{setspace}
\setlength{\parindent}{0in}
\usepackage{physics}
\usepackage{tikz}
\usepackage{tikz-3dplot}
\usepackage[outline]{contour} % glow around text
\usepackage{xcolor}
\usepackage{float}
\usepackage{makeidx}
\usepackage{fancyhdr}
\usepackage{pgfplots}
\usepackage{amsmath}
\pgfplotsset{compat=1.18}
\usepackage{caption}
\usepackage[english,catalan]{babel}
\setlength{\parskip}{11pt}
\usepackage{xcolor}
\usepackage{listings}


\title{\Huge\bfseries Cor:Cor \\[1ex] \Large}

\author{\begin{tabular}{c}
\textbf{GRUP C3} \\
Isaac) \\
Marcel López Freixes (1668323) \\
Eira \\
Núria Castillo Ariño (1669145)
\end{tabular}}

\date{DIA???}

\begin{document}

\maketitle

\section{Introducció}


Tenim 

\begin{equation}
    c_v \rho \frac{\partial T}{\partial t} = \nabla \cdot (\kappa \nabla T) + P_{\text{ext}}
\label{eq: eq_inicial}
\end{equation}


Si simplifiquem el sistema un espai cilíndricament simètric, ens queda


\begin{equation}
    c_v \rho \frac{\partial T}{\partial t} = \frac{\partial^2 T}{\partial x^2} + P_{\text{ext}}
\label{eq: eq_inicial_cilindriques}
\end{equation}
\section{normalització}

Dividint l'eq. \eqref{eq: eq_inicial_cilindriques} per $c_v\rho$ i definint $\lambda=\frac{P_ext}{\rho c_v}$, queda

\begin{equation}
    \frac{\partial T}{\partial t} = \alpha \frac{\partial^2 T}{\partial x^2} + \lambda \implies \frac{1}{\lambda}\frac{\partial T}{\partial t} = \frac{\alpha}{\lambda} \frac{\partial^2 T}{\partial x^2} + 1
\label{eq: eq_entre_rhocv}
\end{equation}

on $\alpha=\frac{k}{\rho c_v}$ és la difusivitat tèrmica.

Si definim $\hat{x}=\frac{x}{L}$, $\hat{T}=T \frac{\alpha}{\lambda L^2}$ i $\hat{t}=t\frac{\alpha}{L^2}$, aleshores

\begin{equation}
    \frac{\partial \hat{T}}{\partial \hat{t}} = \frac{\partial^2 \hat{T}}{\partial \hat{x}^2} + 1
\label{eq: eq_normalitzada}
\end{equation}

\section{Solució analítica}


Seguint la solució de proposada a l'annex del guió amb $q(x,t)=1$ i $\beta=\hat{T}(T_c)$, queda que



\begin{equation}
    f(\hat{x}, \hat{t}) = \beta + \frac{4}{\pi^3} \sum_{n=1}^{\infty} \frac{1 - e^{-(2n-1)^2 \pi^2 \hat{t}}}{(2n-1)^3} \sin\big((2n-1)\pi \hat{x})
\label{eq: analitica}
\end{equation}


\section{Solucio numèrica: Euler explicit}

Per aplicar el mètode numèric d'euler explicit


\end{document}